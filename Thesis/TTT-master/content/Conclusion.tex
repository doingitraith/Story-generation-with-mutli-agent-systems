\chapter{Conclusion}
In the world of consumer media, creators are constantly looking for new ways to tell stories. Since video games are inherently interactive, they provide completely new and innovative ways to experience and create stories. Dynamic storytelling is a relatively new area in game design and game development that focuses on creating satisfying narrative experiences without the need for extensive writing or the previous creation of a story. Dynamic storytelling is an umbrella term for different approaches to achieve this goal such as artificial intelligence to drive the choices of NPCs, complex systems of interlocking game rules that create unexpected outcomes with narrative implications or small bits of previously written narrative content that are rearranged and slightly changed to form new narratives. Using emerging technology is beneficial to exploring these new ways.\\
This thesis presents a novel game mechanic that supports dynamic storytelling by allowing NPCs to exchange game state information and allowing the player to retrieve and introduce pieces of information to the system. The developed prototype game, applies techniques from multiple disciplines of computer science in order to implement the designed game mechanic. Concepts from \textit{Emergence Games}, \textit{Communication Theory}, \textit{Multi-Agent Systems}, \textit{State Machines}, \textit{Consensus Protocols}, \textit{MapReduce} and \textit{Inference Engines} are brought together to create systems that interact on a deep and complex way to allow the NPCs to make decisions and act on them. This approach shows potential that encourages the further development of this game mechanic.\\
I explain the theories and the research behind these techniques and provide the necessary information to understand the implementation of the prototype. This thesis touches on many of the topics that where part of my syllabus both the Bachelor and the Master program of Games Engineering and as such provide a good throughline of the learned theories and concepts. I then give a detailed overview of the different components and systems that work together in order for the prototype to reflect the game mechanic. Sometimes I used only aspects of a technique or the basis of a concept to apply in the prototype but the underlying thoughts are important for the game mechanic once put together.\\
Finally, I compare the developed prototype game to other games that apply different methods of dynamic storytelling and classify those games according to important criteria of dynamic storytelling. The developed prototype achieves to provide interesting and varied narrative scenarios depending on the actions the player executes in the game. There are however, several shortcomings that hinder the prototype to be more immersive in its narrative implications and more varied in its gameplay options. FI suggest areas where further work is most promising and where basic groundwork is already present in the prototype to expand upon.\\
To conclude, there is a lot of untapped potential in the area of dynamic storytelling. This novel approach to leverage information exchange to a core game mechanic to support narrative gameplay, is a further step into this area.