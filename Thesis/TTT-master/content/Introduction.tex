\chapter{Introduction}
Video game players today ask for narrative-focused games more and more often. A report from WePC.com in which 73.55\% of people who play video games prefer single-player games over multi-player games indicates this fact~\cite{WePC2021}. This is, however, not to say that multi-player games can not focus on the story. It is common, though, for single-player games to have a stronger focus on narrative. WePC.com further notes that gamers claim to play more video games because of the COVID-19 pandemic. Between March 23 and June 3rd, 2020, the number of players increased by 46\% in the United States, 41\% in France, 28\% in the United Kingdom, and 23\% in Germany.~\cite{WePC2021}\\
The rise in interest in single-player games is also backed by Sony Interactive Entertainment. With 46\% of the console market, Sony is the leader of a \$58.6 billion industry segment.~\cite{Dealessandri2021} A supposedly leaked document mentioned in a Vice.com article reports that the Sony console's internal tracking tools found that PlayStation users spend more time offline than online regularly~\cite{Klepek2020}. Finally, in an interview from March 2020, the head of PlayStation's Worldwide Studios, Hermen Hulst, says that Sony is "very committed to quality exclusives. And to strong narrative-driven, single-player games."~\cite{Shuman2021} PlayStation Studios consists of 14 individual studios that focus on narrative-driven single-player games~\cite{Sony2021}. All these commitments of the console industry leader to storytelling clearly indicate that the demand for games with a narrative focus is present in the video game landscape.\\
Modern narrative games feature stories that rival movies in terms of production value. In May 2018, Eidos Montreal boss David Anfossi said that the game \textit{Shadow of the Tomb Raider}~\cite{tombraider} cost \$75 - \$100 million just for production~\cite{Dring2018}. These more expensive games usually feature a very cinematic and action-oriented storyline with well-known voice and performance actors. They are also usually quite linear in their structure and storytelling. Developers want to ensure that players actually see and experience the expensive set-pieces in the stories, which explains this more linear approach to storytelling.\\
However, would interactive media like video games not profit from the potential of actually interactive storytelling? Players often wish for games to give them many options in their interactions. "I want to be able to do whatever I want" is a request that comes up frequently when talking about open-world games. Acclaimed game designer Sid Meier in his 2012 talk at the Game Developer Conference, describes games as a "series of interesting decisions". He explains that by limiting and carefully crafting the actions a player can take, the gameplay becomes more compelling~\cite{Dring2018}.\\
So the argument can be made that "being able to do whatever I want" could actually be detrimental to the game experience. This argument can be extended into the story. Authored stories often follow the rules of their respective culture's storytelling traditions. We perceive stories that deal with culturally relatable topics and follow familiar structural rules more favorably.~\cite{Cooney2017}\\
However, would players be able to make interesting narrative decisions naturally, given the option by a game that does not restrict interactions to an authored story? Does a video game story require an author at all? Can we  develop systems that take on that role? Could a game mechanic provide all necessary elements to satisfy narrative requirements? This thesis aims to explore not only what those requirements are and how they contribute to game stories, but also introduces a framework for game mechanics that allow the utilization of world state information for narrative purposes.\\
A story can be abstracted as numerous pieces of world information that are exposed to the person experiencing the story, in this case, the player. If we have a story that goes like "The knight slays the dragon." We know that the world consists of a knight, a dragon, and the act of slaying the former. These pieces of information about the actors and acts in the world are then framed into a narrative structure - the story.\\
In this thesis, I developed a game mechanic based on modeling this kind of information into objects that exist in the game world. They are created by interacting with the world and are perceivable by agents in the system next to the player. By allowing the non-player agents to react to newly learned information, the model creates an interplay among all agents in the system. The resulting game mechanic then is based on retrieving relevant information about characters or other objects in the game world and introducing new information as the player. This mechanic could even be used to create factually wrong information and thus allow the concept of lying to agents.\\
For the prototype game that was also developed as part of this thesis, I chose the following setup to showcase the game mechanic. The player controls a knight in a freely traversable 3D environment that represents a medieval kingdom. There are several key locations like a castle, a village, or a farm that are inhabited by other \textit{non-player characters} (NPC) that represent the agents of the multi-agent system. The NPCs move around the kingdom and execute different tasks, which bring them into contact with each other. When they interact, they will also exchange information and thus learn new things about the game state. The player can also interact with characters and items and learn pieces of information. They can also engage in a conversation with NPCs and ask for or give out specific information. By adding quests to the game, a narrative framing exists that the player can use to motivate their actions.\\
Elevating the exchange of information in multi-agent systems to a core game mechanic and allowing players to exploit the information flow is a novel system that not many games support today. It is also a step towards creating an actual interactive story because it allows for genuinely influencing the game world but still enabling developers to define rules for how agents and the environment reacts and thus enforcing a narrative structure.