\chapter{Introduction}
Narrative heavy video games are ever more asked for by players. This is indicated by a report from WePC.com in which 73.55\% of people who play video games prefer single-player games over multi-player games~\cite{WePC2021}. This is, however, not to say that multi-player games can not focus on story. It is common though for single-player games to have a stronger focus on narrative. WePC.com further notes that gamers claim to play more video games because of the COVID-19 pandemic. Between March 23 and June 3rd 2020, the number of players has increased by 46\% in the United States, 41\% in France, 28\% in the United Kingdom and 23\% in Germany.~\cite{WePC2021}\\
The rise in interest in single-player games is also backed by Sony Interactive Entertainment. With 46\% of the console market, Sony is the leader of a \$58.6 billion industry segment.~\cite{Dealessandri2021} A supposedly leaked document mentioned in a Vice.com article reports, that the Sony consoles internal tracking tools found that PlayStation users spend more time offline than online on a regular basis~\cite{Klepek2020}. Finally, in an interview from March 2020, the head of PlayStation's Worldwide Studios, Hermen Hulst says that Sony is "very committed to quality exclusives. And to strong narrative-driven, single-player games."~\cite{Shuman2021} PlayStation Studios conists of 14 individual studios that focus on narrative driven single-player games~\cite{Sony2021}. All these commitments of the console industry leader to storytelling is a clear indicator, that the demand for games with narrative focus is present in the video game landscape.\\
Modern narrative games feature stories that rival movies in terms of production value. In May 2018, Eidos Montreal boss David Anfossi said, that the game Shadow of the Tomb Raider cost \$75 - \$100 million just for production~\cite{Dring2018}. These more expensive games feature usually a very cinematic and action oriented storyline with well-known voice and performance actors. They are also usually quite linear in their structure and storytelling. Developers want to ensure that players actually see and experience the expensive set-pieces in the stories  which explains this more linear approach to storytelling.\\
But would interactive media like video games not profit from the potential of actually interactive storytelling? Players often wish for games to give them a lot of options in their interactions. "I want to be able to do whatever I want" is a request that comes up frequently when talking about open world games. Acclaimed game designer Sid Meier in his 2012 talk at the Game Developer Conference describes games as a "series of interesting decisions". He explains that through limiting and carefully crafting the actions a player can take, the gameplay becomes more compelling.~\cite{Dring2018}. So the argument can be made that "being able to do whatever I want" could actually be detrimental to the game experience. This argument can be extended into the story. Authored stories often follow rules of their respective culture's storytelling traditions. We perceive stories that deal with culturally relatable topics and follow familiar structural rules, more favorably.~\cite{Cooney2017}\\
But would players be able to make interesting narrative decisions naturally, given the option by a game that does not restrict interactions to an authored story? Does a video game story require an author at all? Can we  develop systems that take on that role? Could a game mechanic provide all necessary elements to satisfy narrative requirements? This thesis aims to explore not only what those requirements are and how they contribute to game stories, but also introduces a framework for game mechanics that allow the utilization of world state information for narrative purposes.\\
A story can be abstracted as numerous pieces of world information that are exposed to the person experiencing the story, in this case the player. If we have story that goes like "The knight slays the dragon." We now that the world consists of a knight, a dragon and the act of slaying the former. These pieces of information about the actors and acts in the world are then framed into a narrative structure - the story.\\
In this thesis, I developed a game mechanic based on modeling this kind of information into objects that exist in the game world. They are created by interacting with the world and are perceivable by agents that exist in the system next to the player. By allowing the non-player agents to react to newly learned information, the model creates an interplay among all agents in the system. The resulting game mechanic then is based on retrieving relevant information about characters or other objects in the game world and introducing new information as the player. This could even be used to create factually wrong information and thus allow the concept of lying to agents.\\
Elevating the exchange of information in multi-agent systems to a core game mechanic, and allowing players to exploit the information flow is a novel system, that not many games support today. It is also, a step towards creating actual interactive story, because it allows for truly influencing the game world but still enabling developers to define rules for how agents and the environment reacts and thus enforcing a narrative structure.