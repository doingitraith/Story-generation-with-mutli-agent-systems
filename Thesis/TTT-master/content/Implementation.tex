\chapter{Implementation} % 25 pages
The following chapter describes the prototype game I created as part of this thesis. I explain how I applied the concepts and research introduced in the previous chapter. First I will introduce the \textit{Information Model}, that I created to define information in the game. Then I will go into detail about the \textit{agents}, that inhabit and exist in the game world. After that, I will describe the \textit{game world} itself followed by a \textit{gameplay} description. Then I will focus on the \textit{dialogue feature} and how information can be retrieved and introduced into the system. Following that I will give an explanation of how narrative is introduced using \textit{quests}. Additionally, I will provide the concrete implementations of both the information \textit{heuristic} and the \textit{consensus protocol}. The chapter will be closed off, by discussing the \textit{narrative implications} of the described gameplay.

\section{Information Model}
At the center of the game mechanic created for this thesis, is the \textit{Information Model}. I propose a conceptional model for information objects that encapsulate statements about semantic and factual game world data (see \ref{section:info}). The statements encapsulated in an \verb|Information| object is primarily characterized by its \textit{Verb}. The verb is the main indicator of the assertion of the information content. Every information object also contains a \textit{Subject} which denotes who or what the information statement is concerned with. The third part of the information is dependent on the verb and can either be an \textit{Object}, an \textit{Adjective} or a \textit{Location}.
Thus every information object $I$ consists of a triple of information elements:
\begin{center}
	$I \coloneqq \{s, v, x\}$ 
\end{center}
Where $s$ is the subject, $v$ is the verb and $x \in \{\textit{Object, Adjective, Location}\}$ depending on the verb. Figure~\ref{fig:informationCD} shows the class diagram for the \verb|Information| class. Although the elements of an \verb|Information| object refer to actual game objects in the game world, this link is purposely severed when an \verb|Information| object is created. This is done, so that the information statement is actually just that and has no exploitable connection to other objects anymore. Thus the \textit{Information Model} and the programming interface remain decoupled which allows for a more accurate simulation.
\begin{figure}
	\centering
	\includegraphics[width=0.3\textwidth]{InformationCD}
	\caption{Inheritance diagram of the Information class}
	\label{fig:informationCD}
\end{figure}
\subsection{Verbs}
Verbs in the \textit{Information Model} come in three types:
\begin{itemize}
	\item \textbf{IS}: Creates information statements about the \textit{state} of a subject of the form "\textit{Subject} \verb|IS| \textit{Adjective}".
	\item \textbf{HAS}: Creates information statements about a possessive relationship between subject and object of the form "\textit{Subject} \verb|HAS| \textit{Object}".
	\item \textbf{AT}: Creates information statements indicating the location of a subject of the form "\textit{Subject} \verb|AT| \textit{Location}". The \verb|AT| type presents a special form of the \verb|IS| type, but in the context of a traversable 3D world, locations represents such an important piece of state, that its own information type is justified.
\end{itemize}
\textit{Verbs} are instances of the \verb|Verb| enumeration. The constructors in the \verb|Information| class are designed in such a way, that the verb is set by the constructor and there exist multiple overloads to create the different types of information objects (Listing \ref{listing:information}).
\begin{lstlisting}[
	caption={Information class constructors},
	label={listing:information}
	]
	public Information(Agent agent, Item object)
		=> (Subject, Verb, Object, Adjective, Location, Not) =
		(agent.InformationSubject, InformationVerb.Has, object.InformationSubject, null, null, false);
	
	public Information(WorldObject subject, InformationAdjective informationAdjective)
		=> (Subject, Verb, Object, Adjective, Location, Not) =
		(subject.InformationSubject, InformationVerb.Is, null, informationAdjective, null, false);
	
	public Information(WorldObject subject, InformationLocation informationLocation)
		=> (Subject, Verb, Object, Adjective, Location, Not) =
		(subject.InformationSubject, InformationVerb.At, null, null, informationLocation, false);
\end{lstlisting}
This way \verb|Information| objects can be created with only the content in mind, and the constructor sets up the object correctly for usage.
\subsection{Subjects}
As mentioned, every piece of information contains a \textit{subject}. They represent the concerning entity of an information statement. A \textit{subject} is an instance of the \verb|InformationSubject| class that conceptually either represents an \verb|Agent| or an \verb|Item|, both of which inherit from the \verb|WorldObject| class.\\
What exact type of object a \textit{subject} is in a given \verb|Information| object, is again dependent on the \textit{verb}. For example there can be no information of the form "\textit{Item} \verb|HAS| \textit{Agent}" because an item cannot own an agent. The constructors for \verb|Information| objects make sure that no erroneous object can be created.\\
\textit{Subjects} consist  of a textual name that denotes the \textit{subject} and two boolean values that state whether the \textit{subject} is an agent or not and whether it is a unique object, which is important for items (Figure~\ref{fig:subjectCD}).
\begin{figure}
	\centering
	\includegraphics[width=0.6\textwidth]{SubjectCD}
	\caption{Inheritance diagram of the InformationSubject class}
	\label{fig:subjectCD}
\end{figure}
\textit{Subjects} also have a \verb|Mutation| object attached that is responsible for decaying the accuracy of the \textit{subject}. I give a detailed explanation of mutations in Section~\ref{section:mutation}.
\subsection{Adjectives}
An \textit{adjective} represents a detail or characteristic of a \textit{subject's} state. The can be arbitrarily defined by a developer. They represent the properties of a \textit{subject}. \textit{Adjectives} are instances of the \verb|InformationAdjective| class (Figure~\ref{fig:adjectiveCD}). In the \textit{Information Model} there are two differentiable types of \verb|InformationAdjective| objects:
\begin{itemize}
	\item \verb|InformationProperty|: Represents factual information about a \textit{subject}. For example "\textit{Agent} \verb|IS| \textit{alive}." is a factual piece of information that can be either true or false.
	\item \verb|InformationOpinion|: Represents statements about a \textit{subject's} state that are influenced by information holder's relationship to the \textit{subject}. For example the statement "\textit{Agent} \verb|IS| \textit{dangerous}." can be true for one agent but false for another.
\end{itemize}
This distinction is introduced so that conceptually there can be a difference between facts and opinions that someone has about any \textit{subject}.\\•
An \verb|InformationAdjective| consists of the textual \textit{Characteristic} and a list of \textit{contradictions}. The contradictions are other \textit{adjectives} that are conflicting with the \textit{adjective} they are attached to. A simple example would be the two information statements  "\textit{Agent} \verb|IS| \textit{alive}." and  "\textit{Agent} \verb|IS| \textit{dead}.". Both cannot be true at the same time. Therefore, each \textit{adjective's} contradictions list contains the respective other. When an agent learns a new information that contains an \textit{adjective}, they check the \textit{adjective's} contradictions and resolve any conflicts.\\
At the start of the game, an initial list of \textit{adjectives} that exist in the world is created. After that, the contradictions lists are created and attached to each \textit{adjective}.
\begin{figure}
	\centering
	\includegraphics[width=0.8\textwidth]{AdjectiveCD}
	\caption{Inheritance diagram of the InformationAdjective class}
	\label{fig:adjectiveCD}
\end{figure}
\subsection{Locations}
As mentioned before, a \textit{location} is a special kind of \textit{object} for the \verb|AT| information type. In a 3D game world that is driven by a multi-agent system, agents need to move around. For that the \verb|InformationLocation| class attaches a position in the world to the information so that agents can retrieve an actual location from the information statement.\\
A \textit{location} also has a \verb|Mutation| object attached to it, so its accuracy can be reduced. Other than that, the \verb|InformationLocation| has only a textual name and the 3D position as a public interface.
\begin{figure}
	\centering
	\includegraphics[width=0.6\textwidth]{LocationCD}
	\caption{Inheritance diagram of the InformationLocation class}
	\label{fig:locationCD}
\end{figure}
\subsection{Mutation}
\label{section:mutation}
To simulate the decay of information and agents forgetting details, the \textit{Information Model} implements a \textit{Mutation} functionality. \verb|InformationSubject| and \verb|InformationLocation| objects have an object of the \verb|Mutation| class. These represent a hierarchy of less precise information values. Every \verb|Mutation| object has a reference \verb|ParentMutation| that points to a \verb|Mutation| object that holds the next less precise information value.\\
As an example, we assume the \verb|InformationLocation| $l$ with the name value "\textit{Village Square}". The \verb|Mutation| object can point to a less precise information value "\textit{Village}" which in turn points to an information value "\textit{Somewher in the South}". This way when $l$ is mutated once, its name value will be "\textit{Village}" and if mutated again will decay to "\textit{Somewhere in the South}" where it will remain.\\
This mechanic allows for agents to slowly forget about details in their information statements which makes the collection of memories more chaotic over time.\\
The values for the mutation hierarchy are set by the developer before the game starts.
\section{Agents}
In any multi-agent system, the central component are of course the agents. In the prototype game, those agents represent the NPCs who live in the game world or the player (Figure~\ref{fig:agentCD}). The only difference is that the player agent is controlled by the player instead of the game. But for the NPCs, there is no difference between the player agent or any other NPC. In the following I will generally talk about NPCs as agents for better understanding of the conceptual implications in a multi-agent system. Agents move around in the world to perform actions, exchange information and change their state depending on what they learn.
\begin{figure}
	\centering
	\includegraphics[width=0.6\textwidth]{AgentCD}
	\caption{Inheritance diagram of the Agent class}
	\label{fig:agentCD}
\end{figure}
The \verb|WorldObject| class is the parent class for both \verb|Agent| and \verb|Item| classes. I represents physical entities that exist as part of the multi-agent system. \verb|WorldObject| itself is a child class of Unity's \verb|MonoBehaviour| class which is the base class of all game objects managed by the Unity engine.\\
All \verb|Agent| instances inherit the members of their parent classes and thus have a textual identifier \textit{Name}, a \textit{Location} to denote the object's current location, exposed properties to set up the \textit{InformationSubject} and an integer value \textit{WorldImportance}. This value is important for the information heuristic for believability which is described in section %TODO add ref to heuristic.
\textit{WorldImportance} indicates the initial interest a piece of information gets through the involved object. This interest can be positive or negative. A piece of information about the castle might have high positive value, since the castle is an important place in the kingdom. Information about an enemy bandit however, will have a negative number, as people are afraid of the bandits. These values should be an objective measure of an object's interest value. Subjective views from the agents are taken into account when the heuristic is calculated.\\
The important properties of the \verb|Agent| class are the \textit{Inventory} which stores items in the agent's possession, \textit{Acquaintances} contains other agents that have interacted with the object, \textit{ImportantPeople} are agents that have a close relationship with the object like friends or loved ones and \textit{Quests} which is a list of objectives the agent has. Through the \textit{ImportantPeople} property, certain agents can be treated differently and more favorably than others. This allows the introduction of a social network component into the multi-agent system.\\
The boolean values \textit{IsSeeing} and \textit{IsHearing} allow to determine what kind of information the agent is able to "sense" respectively.\\
The heart of the game mechanic is the \textit{Memory} property. It is an \verb|InformationManager| object. As the name implies, it stores, processes and manages all the incoming pieces of information.\\
The different actions an agent can execute are also defined here. This includes interacting with other agents, picking up and dropping items or attacking agents.
\subsection{NPC}
The \verb|NPC| class is a child class of the \verb|Agent| class and defines additional properties and functions that are unique only for NPCs.\\
This includes a \textit{Routine} property which is a list of behaviors the agent will execute one after another to simulate a daily life of a human. When they reached the end of their routine, they will start again from the beginning.
\subsection{Player}
In the scope of this thesis the \verb|Player| class behaves very similar to other agents. It has components attached that enable input to control the player agent. It needs its own class, so that developers can differentiate between NPCs and the player. This is necessary as the agents don't might want to behave differently when interacting with the player agent to expose more information to the actual player in front of the screen. For example when a dialogue is started, an NPC will not simply send information and continue with its routine, but it will pause and the dialogue interface will be opened where the information exchanged is controlled by the player.
\subsection{Behavior}
In order for the intelligent agents to move, interact and follow their objectives, I defined multiple behaviors that allow them to perform actions (Figure~\ref{fig:behaviorCD}). The following behavior classes exist:
\begin{itemize}
	\item \verb|ExchangeInformationBehavior|:
	\item \verb|SendInformationBehavior|:
	\item \verb|WalkBehavior|:
	\item \verb|TalkBehavor|:
	\item \verb|WaitBehavior|:
	\item \verb|PickUpBehavior|:
	\item \verb|DropBehavior|:
\end{itemize}
\begin{figure}
	\centering
	\includegraphics[width=1.2\textwidth]{BehaviorCD}
	\caption{Inheritance diagram of the AgentBehavior class}
	\label{fig:behaviorCD}
\end{figure}
\subsection{Information Manager}
% Agents 4
\section{Game World}
% Game World 2
	% Technology
\section{Gameplay}
% Gameplay 2
\section{Dialogue}
\subsection{Questions}
% Questions 2
\subsection{Statements}
% Statements 2
\section{Quests}
% Quests 2
\section{Inference}
% Inference 2
\section{Heuristic}
% Heuristic 2
\section{Consensus}
% Consensus 2
\section{Narrative}
% Narrative