\chapter{Discussion} % 4 pages
This chapter puts the results of the prototype game into perspective. I will discuss which parts of the declared goals have been achieved and which ones need further work. Furthermore, the challenges for adapting the presented approach in the given scope are explained. Finally, potential future research and work areas are discussed.
\section{Achievements}
The goal of this thesis is to create a novel game mechanic that uses the exchange of game state information to support narrative content creation while still allowing for narrative framing from designers.\\
I achieved this by simulating the information exchange by modeling a multi-agent system of NPCs. The resulting prototype shows this core mechanic quite proficiently in regards to the simulation of information flow. The agents move around the game world and exchange information that they learn with one another. They tend to share more "interesting" information. The quality of "interest" of information is calculated by taking the source of information into account as well as the probability of the information be- ing true. This creates a much-needed context for the information exchange that changes for each agent as they meet more other agents and learn new information.\\
There is also a social aspect to the exchange of information. Each character has a set of other characters that form their "circle of influence". This is a group of characters who will be trusted more than other agents, which means that information coming from them will be seen as more believable. This way, friends, family, or other influential relationships are modeled in the system as well.\\
When it comes to the narrative capabilities of the prototype, I have described in previous chapters how it allows for different playstyles that result in vastly different narrative structures. By carefully crafting quests in an objective-oriented manner, it is possible to leave the methods of completing a quest quite open to the player.\\
This means that the game mechanic can achieve the goal of supporting or even creating storylines that are not previously defined by a designer but originate in the player’s actions in the decision during the game.\\
From a technological point of view, the prototype uses techniques from several areas of computer science and applies them in order to support the desired outcome. It is even prepared in such a way that it could distribute processing-intensive calculations to other processes in order to optimize performance. That is something that not many games use yet since it would require a separate infrastructure of support computers that need to be available in order to play the game.\\
Other methods used in the prototype are consensus protocols, state machine modeling, and inference engines. The prototype takes aspects or concepts from each of them and uses them to support the game mechanic. The result is a prototype that is accessible to somewhat experienced players of video games that is capable of showing the game mechanic in a robust and consistent way.
\section{Shortcomings}
Not all desired features or capabilities found their way into the prototype game due to the scope of the thesis or other challenges. In this section, I want to describe the aspects that still require additional work.
\subsection{NPC Reactions}
Maybe the most crucial missing feature of the prototype is the actual reactions of agents to learning information. NPCs do adapt their internal state and evaluate other characters and pieces of information differently with the information they have and learn, but they do not adapt their behavior in any significant way. This could truly elevate the narrative capabilities of the prototype to the next level because it would allow even more dynamic behavior and situations that carry narrative implications. It would have required extensive additions to the \textit{Information Model} and to the agent AI to do so, and sadly this was not possible in the scope of this thesis.
\subsection{Adaption of the NPC Routine}
One approach to introduce reactions to the agent behavior would be to implement insertions or other manipulations to an NPC’s routine. The routine is the list of behaviors an agent executes in sequence in the game. Allowing changes to that list based on learned information would introduce more dynamic behavior as well while allowing the NPC to return to their original stable routine after an inserted behavior has been executed.
\subsection{NPC Actions}
Generally speaking, in order for NPCs to more dynamically react to changes in the game state or to new information, they need to be able to perform more varied actions. The prototype is implemented in a way to easily create and add new behaviors for NPCs, but of course, the present version would have profited from more available behaviors. Examples could include fighting, trading, or performing professional jobs. Modeling complex agents was, however, not the focus of this thesis which is why additional behaviors were not extensively developed.
\subsection{Conversation}
One specific NPC behavior that would instantly make the agents seem more active and goal-driven as compared to the current state would be the ability to start a conversation on their part. With the already existing system to ask for or introduce specific information, NPCs could walk up to another player and ask them for a specific piece of information. This could also be used to make NPCs able to lie to others by telling them semantically correct but factually wrong information actively. Tied to a more capable NPC AI, this would make them into more active participants in the game.\\
Additionally to NPCs not being able to initiate a conversation, the dialogue system right now is also very basic. Information is presented textually in a very raw form that is usually grammatically incorrect yet understandable. Additional work on the dialogue system using, for example, generative grammars could increase both readability and variation in the dialogue lines.
\subsection{Overall Content}
Finally, although the prototype game world is aesthetically pleasing and large enough for exploration to take place, it lacks content to interact with. There are many agents walking around and going after their business which was the focus of the thesis, but when it comes to items to interact with and interactions to perform, the prototype is lacking. It was again the main focus to show the capabilities of the game mechanic in the scope of this thesis, and adding more varied content is something I would gladly explore in the future.
\section{Outlook}
In addition to the described areas that improve and extend existing systems, there are several areas that would require extensive reworks and additions that would make the game more novel and more innovative.
\subsection{NPC Learning}
First on this list would be the ability to learn for NPCs. By that, I mean utilizing machine learning or reinforcement learning techniques to allow them to evaluate information more accurately. The Unity engine contains so-called \textit{Machine Learning Agents} to provide just such capabilities. They use reinforcement learning and intimidation learning to train the agents that can then be used as agents in a game.\\
Adding complexity to the \textit{Information Model} would also enable big improvements for the decision-making processes of the NPCs. I mentioned \textit{Emotion Engines} in Section~\ref{section:emotion}, which would be a worthwhile addition to my own \textit{Information Model}. Extending the social awareness of the NPCs could also yield more character-driven narrative structures when they are interacting.
\subsection{Towards Intelligent Agents}
In order to leverage the more complex decision-making and social capabilities of NPCs, a more complex AI would also be a prospective field of further research. Since the quests in the prototype game are already defined as objective-oriented, work could be done towards allowing the NPCs to follow those goals as well, or even goals of their own. They could be equipped with their own motivations and priorities that change then depending on what they learn or deduce. Truly intelligent agents would very probably yield to unexpected and interesting narrative outcomes and situations and make the game truly "come alive".\\
By then, the game would be best described as a \textit{social simulation} that uses quests to create agency for the player. However, with all the systems interacting on such a deep level, players could also work towards goals of their own design and see how far they can bend the limitations of the simulation.
\subsection{Narrative Awareness}
Finally, to really add to the narrative novelty of this game mechanic, it would be a very promising direction to do research into the area of narrative awareness of the game. By that, I mean to design the AI of the agents and maybe an overseeing game logic entity to steer the parts of the game towards more dramatic decisions. For that to work, stories would have to be formalized in such a way that the game logic could evaluate the current game state and then decide what would need to happen to create a satisfying dramatic situation. This could then be translated into the behavior of single agents who would change their behavior in order to satisfy the dramatic needs of the game. This would go in the direction of creating a virtual \textit{Game Master} that is common in tabletop role-playing games. An entity that controls all the NPCs and events in the game so that an interesting story unfolds while still being able to react to player actions and incorporate them into its own decision-making process.