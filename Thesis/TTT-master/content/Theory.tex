\chapter{Theory} %25 pages
This chapter present the various fields of study, that were encountered or utilized while developing the game mechanic and its implementing prototype game. This includes fundamental story, game and quest theory, information theory, multi-agent systems, consensus protocols, emotion engines and inference engines.

\section{What is a story?}
There exist a multitude of definition on what a story is. Rayfield argues that story is a narrative item that exists throughout all cultures. He concludes that there exist a universal concept of a certain structure that listeners will recognize as a story. He limits this structure by degree of complexity and argues that listeners will only recognize the structure as story within certain minimal and maximal bounds of complexity. These bounds would then be the same across all cultures.~\cite{Rayfield1972} Scheub takes another approach and sees story more as "a means whereby people come to terms with their lives, their past; it is a way of of understanding their relationship within the context of their traditions. It is a means of accessing and valuing history: in the end story \textit{is} history."~\cite{Scheub1998} Lastly the Oxford Dictionary of Literary Terms more formally describes story as a set of events that are selected and arranged in a specific order and told by a narrator. The specific order of the events is called the plot.~\cite{Baldick1996}\\
One thing that most definitions have in common however is that stories are an important tool for humans. Stories help us to interpret and process information and experiences. They enrich subjectively perceived facts and form them into each person's individual truth. This is the \textit{"meaning"} of a story. This "meaning-making" is also part of the psychological process in self-identification and the creation of memories~\cite{Flanagan1992}. Stories are furthermore an important factor in human communication, used as parables and examples to illustrate points. Storytelling was one of the earliest forms of entertainment.\\
When looking into definitions of what a story \textit{is}, the matter quite quickly goes into the area of what a story \textit{does}. I already mentioned how it is a vital tool for the human psyche but there are some more concrete functions that stories fulfill.

\subsection{Functions of Storytelling}
The motifs and contents of stories and the modes of narration are highly culturally individual aspects. The functions these elements serve though, can be found across all cultures. One example is to make one narration more understandable by putting it into relation with another story. This is commonly referred to as a metaphor. Now these relational stories do't have to be imaginary per se. They can be a retelling of events that have actually transpired an help underline the point of the narrator. On the far hand of abstracting stories to make a point is the very careful use of words to find an objective true transpiring of events. This is what happens in courtrooms. It is a retelling of events, but the order and selection is so careful, so meticulous, that an actual fair "true" story might be revealed.~\cite{Rigney1992}\\
As mentioned above stories and storytelling are used to share and interpret experiences. The human brain is evolutionary predisposed to process, store and recall memories in the form of stories~\cite{Wyer2014}. Humans think in narrative structures and mostly remember facts in the form of a story. Facts are smaller versions linked to a larger story, which supports analytical thinking.~\cite{Connelly1990} This makes storytelling such a great tool for teaching as well. There is research about how storytelling is a meaningful teaching method that can be applied in education to encourage the development of caring, empathy, compassion and to develop a deeper cultural understanding.~\cite{Davidson2004} It is not event solely the listening person who is learning from a story. Often the discovery of a personal meaning of a story is only made visible when telling a story. Thus also the storyteller can learn something new.~\cite{Doty2003}\\
This is applied in therapeutic storytelling, where through retelling experiences in story form, the storyteller attempts to better understand their own thoughts and situation. This can be supported by questions from a therapist who carefully steers the storyteller through their narration to pinpoint insights.~\cite{Lawless2001}\\
There are countless situations where we encounter storytelling precisely because we want to share experiences and emotions. Stories are used to inspire and motivate, to manage conflicts, for marketing or for political practice~\cite{Jameson2001}. These are all situations where the intent is separate from the story itself. Where we turn to storytelling because the human brain can process them so effectively. Another aspect is however when we tell stories for the story's sake.


\subsection{The Appeal of Stories}
I have now established that there is a difference between the intention of storytelling itself and utilizing it for another purpose. The difference is that we do not enter a courtroom to tell or listen to a story. We do so to find the truth. The past has simply show us that the careful recounting of events combined with precise inquiry, has proven a good way to do so. So when we do not tell stories as a means of achieving another intention, we also share them just because they are stories. For this we can take on both positions, that of the narrator or the listener. Again, a multitude of situations presents itself for these applications. We listen to bedtime stories that our elders tell us. We tell a funny anecdote to our friends on a night out. Humans have created whole industries around the consumption of stories. We read books to immerse ourselves into stories, we watch movies or shows and we play narrative video games. The fact that so much of our time is willfully spent listening to, watching or interacting with stories, must mean that there is something worthwhile there. The mere consumption of a narration seems to be satisfying in itself.\\
When we hear stories, the brain releases the hormone Oxytocin. This hormone heightens feelings of trust, empathy and compassion. It positively influences social behavior and helps us to feel more connected to others.~\cite{Gottschall2012} Humans are inherently social beings. We want to connect to others around us. We try to create situations and use known cultural signals to connect on a non-verbal level. Humans mimic body language, laugh more in social situation than alone or use physical contact to communicate.~\cite{Frith2007}\\
When we hear a story, the brain is enabled to form connections to the people who listen to the story with us, to the narrator and to the characters in the story. Communication is a shared activity resulting in a transfer of information across brains. Research shows that during successful communication, the brains of both the speaker and the listener shows common, temporally linked, response activities. When we hear a story, our brain mirrors activities in the sensory center of the storyteller.~\cite{Stephens2010} This attempt to sync brain activity is a deeply social connection. It also means, that when we hear an enjoyable story, the brain behaves as if we would experience it ourselves.

\subsection{Stories in Media}
The consumption of stories has led to the creation of huge industries that focus entirely on creating and delivering stories to customers around the world through different kinds of media. That term comes from the Latin word \textit{medium} for "middle" which again stems from the ancient Greek word \textit{méson} for "the middle" or "the public". Today we use the term media as word for "the means of communication".~\cite{Hoffmann2000} In the context of storytelling this could be oral through a present storyteller, audio through a audiobook, visual though a book or e-book, audiovisual through theater plays, movies or series or interactive through games and other interactive media. Oftentimes, when we talk about "the media" today, we refer to an industrialized consumption of content or news.\\
The economic factor of these storytelling media cannot be underestimated. The global Entertainment \& Media Outlook analysis from PriceWaterhouseCoopers reports over US\$~40bn for global box office revenue in cinemas for 2019. Cinema revenue of course took a nosedive because of the COVID-19 pandemic in 2020 and 2021 but are expected to recover and grow by 2023. Video games however have been steadily growing to roughly US\$~62bn for traditional games revenue and almost US\$~92bn for social or casual gaming revenue in 2021. These numbers are also expected to grow in the coming years.~\cite{PwC2021}\\
Storytelling of course works differently and has different requirements in each of these mediums. Books for example allow for a seamless switch of inward and outward perspective of characters or though processes. Movies as an audiovisual medium, are more restricted to an outward perspective and have to use different storytelling techniques to transport the inner feelings of characters. There is also a structural difference to all theses forms of media since there are conventions for runtime for movies or an episode of a show.~\cite{Ryan2004}\\
The requirements for video game stories stem greatly from the fact that there is a distinction between the human player in front of the screen who controls a player character and this character who inhabits and acts in the game world. The story that we experience in the game, the plot points that we traverse was aptly defined with the term of narrative by Abbot. Similarly to Baldick in the Oxford Dictionary of Literary Terms, he says that narrative is the representation of an event or an action or a series of events or actions. He further argues that without an action, only description or exposition remain~\cite{Abbott2020}. This leads to the idea that narrative can only exist when there is a transition from one state into another. This transition of states is then an event or an action. It fits well with how Aristotle described a story as a whole that "[...] has a beginning and middle and end."~\cite{Aristotle2006}. This idea also implies the existence of at least three states in a story, the beginning, the middle and the end in addition to the transitions in between. These transitions would be, according to Abbot, the narrative.

\subsection{Expectation \& Feedback}
Players have certain learned expectaions when controlling a player character in a narrative game. They expect their gameplay actions to faithfully represent the motives of the character on the screen. When I talk about the player character, I refer to the virtual character in the game world as a function of the human player who controls them on-screen. It is the character we follow in the story. In their work about Character and Conflict, Lankoski and Heliö argue that a narrative state transition, an action, is the most important feature from the player’s point of view~\cite{Lankoski2002}. They also argue that character-driven action is very similar to character-driven writing, a similarity that is especially important for story-driven games. They say "[...] instead of writing a story, characters and their needs are designed so that there will be enough action and conflict in the game to make it playable and interesting."~\cite{Lankoski2002} This makes motivation or needs so important for directing action and keeping a game interesting.\\
This idea of actions that make a game playable and interesting is well explained with the definition of agency by Wardrip-Fruin et al. They define agency as a phenomenon involving both player and game. It "occurs when the actions players desire are among those they can take (and vice versa) as supported by an underlying computational model."~\cite{Rohtua2009} They shift the focus of a definition of agency toward a question of how games can evoke desires that can be satisfied by the game systems. Or how said systems can be created. How to manage player expectations and early on explain the game systems so that player expectations and thus player agency can adapt. This finally brings me to the term of the avatar. Klevjer agrees that avatars are "little more than a cursor" to facilitate player agency in the game. They are tools to fulfill the actions that the player wants to perform~\cite{Klevjer2012}. So the term avatar stands on the opposite side of the term player character. One is the tool used by the player to create meaningful actions, and the other is a character in a narrative with goals and motivations. If the game manages to merge the motivations of
the player and the player character, then the game succeeds in creating agency.\\
This makes agency one of the primary requirements for a video game story. The game designers' intent or the target emotional state and the means by which the game designer expects the production team to induce that emotional state in the player is what Callele et al. call \textit{the emotional requirement}. This requirement needs game developers and storytellers to carefully align their intentions. They need to take cultural, spacial and relational aspects of gameplay events and symbols into account to ensure that their intended emotion is transferred to the player.~\cite{Callele2008}\\
Another requirement for game stories is one of more concrete communication. Games convey meaning different to other forms of media. Every game brings with it a specific set of semantics, the game rules. These need to properly communicated to the player in order for them to be able to play it. A player who understands the game's rules, knows the available actions they can take. This allows them to adjust their expectations to their character and the story. If the story presents the player character as a legendary warrior capable of defeating any enemy, but the game gives me no option to engage in combat, the player will possibly not understand how the reputation of the character came to be in the first place. Likewise a pacifistic character with an arsenal at their disposal will create a similar conflict of intentions between the game and the player. Rule conventions also sometimes dictate feedback that the player expects. An example would be, if the player triggers the input for a \textit{move right} action, they expect the character to move to the right on the screen. This creates an interplay between expectations and feedback that comprises both gameplay and narrative. Symbols come with certain expectations for actions to be available to the player and events triggered by actions come with certain expectations of feedback to be received from the game.

\subsection{The optimal Experience}
A synchronization between expectation and feedback is also something Csikszentmihalyi describes when he talks about the \textit{optimal experience}. He describes incidents when a sense of exhilaration and of deep enjoyment is felt. These moments are often highly active and not receptive or relaxing times. They occur especially when we engage in a voluntary endeavor to do something difficult or worthwhile. Optimal experiences result when there is order in the consciousness. This happens when we are focused on a realistic set of goals with our skills matching the opportunities for action. It can be easily seen how this translates very well to gameplay in video games. Goals allow people to concentrate attention on the task at hand, with other things fading into the background. The key element of an optimal experience is that it is an end in itself. It may be taken on for other reasons initially but the activity soon will become intrinsically rewarding. It has to be autotelic. The term is formed from the Greek words for \textit{self} and \textit{goal} respectively and means, the activity has an end or purpose in itself.~\cite{Csikszentmihalyi1990}\\
I have established earlier that experiencing stories is in fact an autotelic activity. With the science of optimal experience, we can assume that a game mechanic that could satisfy narrative requirements would indeed create a very rewarding gameplay experience.

\subsection{Quests}
Narrative focused games often feature a very specific concept for compartmentalizing their stories - the \textit{quest}. The word has Latin origin in the word \textit{quaesta} for inquire or search. In literary it has long been known the describe the difficult, often symbolic or allegoric journey towards a goal. It plays a central role in Joseph Campbell's \textit{"Hero's Journey"}, where he says, "[...]the hero sets forth from the world of common day into a land of adventures, tests, and magical rewards. Most times in a quest, the knight in shining armor wins the heart of a beautiful maiden/ princess"~\cite{Campbell2008}. Video games also have been using the term for decades for a concrete isolated set of goals, that marks progression. A quest is a task, a mission. It has a set of goals that need to be completed in order for the quest to be finished.\\
Aarseth defines three basic quest types that can be combined and rearranged to create complex structures:
\begin{itemize}
	\item \textbf{place-oriented}: The quest requires the player to move from a starting position $A$ to a target position $B$. Typically the player will encounter obstacles along the way, but the basic premise remains.
	\item \textbf{time-oriented}: These quests add a time related aspect to the goal. For example: \textit{"Survive for $X$ seconds."} or \textit{"Find three apples in $Y$ seconds."}
	\item \textbf{objective-oriented}: This includes a concrete result to the goal requirement. For example \textit{"Get item $I$ from character $J$."} A specific condition needs to be met for the goal to be finished.
\end{itemize} 
The basic types can be combined, nested, serialized or parallelized in order to create highly complex quest worlds. Common types are serial arrangement of quests into a linear corridor, a semi-open hub and an open landscape.~\cite{Aarseth2005}\\
Open quest-landscapes are often featured in open world role playing games like \textit{The Elder Scrolls V: Sykrim}~\cite{skyrim}. Interestingly enough the hub structure mimics the literary hero's journey quite well. It has a distinct "homebase" from where the hero ventures out into the unknown and dangerous lands to face obstacles, only to return to his home a changed person. This is a structure that has been well researched in many old texts such as the myths and tales centered around King Arthur. Spearing describes this in his work on Sir Gawain and the Green Knight. A knight in search of honor leaves the round table and the comfort of Camelot to seek his fortune. He faces dangers and characters that change him and eventually returns home. He know understands this place of familiarity differently, not because \textit{it} has changed, but \textit{he} has.~\cite{Spearing1994}\\
It has been on purpose, that I wrote of the game and the narrative as two distinct, even if interlinked, entities. I agree with Aarseth that what may resemble narrative structures is actually spatial (go from $A$ to $B$) structures. Games that might appear the most story-like are in fact often reliant on place-oriented quests and very spatially constrained. The challenge for game developers as he puts it is to "\textit{[...] move beyond the constraints of unicursal corridors or multicursal hub structures while keeping the player’s attention on a storyline. And that is no easy task. But perhaps presenting an interesting landscape with challenging quests is enough?}" This is exactly what the prototype of this thesis attempts to do.~\cite{Aarseth2005}


\section{Game Mechanics}
In the previous sections I talked at length about the different requirements for a story in video games and how the narrative relates to the gameplay. In this part I will discuss the topic of game mechanics in general. To talk about story \textit{as} the game mechanic through the exchange of information, I need to provide sufficient definitions in both areas.\\
Sicart describes a suitable formal definition of game mechanics that stems from object-oriented programming, while still keeping many ludological aspects such as the figure of players, or agents, as fundamental to understand how games are designed and played. He does so by purposely distinguish between game mechanics and game rules. The former are the options of interactions available to an agent in the game. The latter are the restrictions under which those options are made available and how they change the game state. Sicart arrives at the definition that game mechanics as "\textit{methods invoked by agents, designed for interaction with the game state."} The first part of this definition is rooted in the paradigm for object oriented programming. This appropriation of terminology is useful because it provides a formalistic perspective to actions performed in information systems like games. This can even lead to the application of modeling languages like UML (Unified Modeling Language) to the description of game systems.~\cite{Sicart2008}\\
In object oriented programming, a method is the actions or behaviors available to a class. Methods are the functions an object has for calculating data or interacting with other systems or objects.~\cite{Weisfeld2008}\\
Following this logic, a game mechanic is the action invoked by an agent to interact with the game world, as constrained by the game rules. \textit{Super Mario 64}~\cite{mario64} usually allows the player to jump. If Mario is standing on quicksand however, he will slowly sink into the ground an jumping repeatedly will only bring him back to the surface. That means a mechanic is limited by the rules that apply to the game world such as physics simulations. Mechanics can also be limited by rules that are applied to mechanics in specific game state contexts, like being only able to attack, when a weapon is equipped.\\
The object oriented approach of invoking methods for interactions, decouples the interaction from actual player input. On this systemic level, Sicart is able to talk just about agents that are interacting with the game world and not necessarily player interactions. Game mechanics can be invoked by any agent in the game, be that human or part of the computer system. Agents using artificial intelligence also have a number of methods available to interact with the game world. How actual player interactions are then mapped to input devices cane be viewed separately.\\
The second part of the definition says that game mechanics are methods "\textit{designed for interaction with the game}". These interactions modify the game state. Game developers design basic mechanics that enable the player to overcome obstacles in the game. Challenges here, describe gameplay situations in which the player has to invest effort to progress further. All challenges are matched with a mechanic: by jumping, Mario can reach a higher up area. By striking with an equipped weapon, a character can damage an enemy. The fact that games are structured as systems with mechanics, rules and challenges is understood as the essential grammar of video games.~\cite{Sicart2008}\\
I have touched upon the fact earlier that there is more to the act of playing a game than just interacting with mechanics constrained by rules. In the act of playing, players will appropriate agency within the game world and behave in unpredictable ways. As mentioned before it is one thing what a designer intents for the player to feel, and another, how players actually interact with the game world. The formal, analytical understanding of mechanics only allows us to design and predict courses of interaction, but not to determine how the game will always be played, or what the outcome of that experience will be. Designers can build rules and challenges with specific emotions in mind that they want to invoke in the player and then create matching mechanics that allow to overcome the challenges and thus create other resulting emotions.\\
This systemic approach makes two methods of analyzing game mechanics possible: For one, it allows to systemically analyze the structure of games in terms of actions available to agents to overcome challenges. The second perspective is the analysis of how actions are mapped onto input devices and how mechanics can be used to create specific emotional experiences in players. Sicart does however point towards gray areas in his definition. Perhaps the most significant is the categorical distinction between rules and mechanics. It part of ongoing scientific discourse if mechanics are in fact a subset of rules. He argues, however, that rules are normative, while mechanics are performative and that therefore this ontological distinction can be beneficial for the analysis of computer games. Game studies history shows that there is no dominant definition of key concepts like rules or mechanics. It is in fact the core message of Sicart's work that it is possible and useful to understand game mechanics as different from game rules, and in that understanding, it can more clearly described how games can be designed to affect players in new and innovative ways.~\cite{Sicart2008}
% Game Mechanic - 3 pages
	% emergent
	% multiplicative game design
\section{Concretizing Information}
% Information - 3 pages
	% Types of Data
	% Mutation
\section{Multi-Agent Systems}
% Agents - 3 pages
	% Games as multi agent systems
\section{Consensus}
% Consensus - 3 pages
	% Quorum
	% Proof of Personhood
\section{A Heuristic for Information}
% Heuristic for story relevant information 3 - pages
	% MapReduce
\section{Related Work}
\subsection{Emotion Engines}
% Related Work - Emotion Engines - 1 pages
\section{Inference engines}
% Inference engines - 3 pages