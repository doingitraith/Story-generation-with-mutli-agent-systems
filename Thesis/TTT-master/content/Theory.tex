\chapter{Theory} %25 pages
This chapter present the various fields of study, that were encountered or utilized while developing the game mechanic and its implementing prototype game. This includes fundamental story, game and quest theory, information theory, multi-agent systems, consensus protocols, emotion engines and inference engines.

\section{What is a story?}
There exist a multitude of definition on what a story is. Rayfield argues that story is a narrative item that exists throughout all cultures. He concludes that there exist a universal concept of a certain structure that listeners will recognize as a story. He limits this structure by degree of complexity and argues that listeners will only recognize the structure as story within certain minimal and maximal bounds of complexity. These bounds would then be the same across all cultures.~\cite{Rayfield1972} Scheub takes another approach and sees story more as "a means whereby people come to terms with their lives, their past; it is a way of of understanding their relationship within the context of their traditions. It is a means of accessing and valuing history: in the end story \textit{is} history."~\cite{Scheub1998} Lastly the Oxford Dictionary of Literary Terms more formally describes story as a set of events that are selected and arranged in a specific order and told by a narrator. The specific order of the events is called the plot.~\cite{Baldick1996} One thing that most definitions have in common however is that stories are an important tool for humans. Stories help us to interpret and process information and experiences. They enrich subjectively perceived facts and form them into each person's individual truth. This is the \textit{"meaning"} of a story. This "meaning-making" is also part of the psychological process in self-identification and the creation of memories~\cite{Flanagan1992}. Stories are furthermore an important factor in human communication, used as parables and examples to illustrate points. Storytelling was one of the earliest forms of entertainment.\\
When looking into definitions of what a story \textit{is}, the matter quite quickly goes into the area of what a story \textit{does}. I already mentioned how it is a vital tool for the human psyche but there are some more concrete functions that stories fulfill.

\subsection{Functions of Storytelling}
The motifs and contents of stories and the modes of narration are highly culturally individual aspects. The functions these elements serve though, can be found across all cultures. One example is to make one narration more understandable by putting it into relation with another story. This is commonly referred to as a metaphor. Now these relational stories do't have to be imaginary per se. They can be a retelling of events that have actually transpired an help underline the point of the narrator. On the far hand of abstracting stories to make a point is the very careful use of words to find an objective true transpiring of events. This is what happens in courtrooms. It is a retelling of events, but the order and selection is so careful, so meticulous, that an actual fair "true" story might be revealed.~\cite{Rigney1992}\\
As mentioned above stories and storytelling are used to share and interpret experiences. The human brain is evolutionary predisposed to process, store and recall memories in the form of stories~\cite{Wyer2014}. Humans think in narrative structures and mostly remember facts in the form of a story. Facts are smaller versions linked to a larger story, which supports analytical thinking.~\cite{Connelly1990} This makes storytelling such a great tool for teaching as well. There is research about how storytelling is a meaningful teaching method that can be applied in education to encourage the development of caring, empathy, compassion and to develop a deeper cultural understanding.~\cite{Davidson2004} It is not event solely the listening person who is learning from a story. Often the discovery of a personal meaning of a story is only made visible when telling a story. Thus also the storyteller can learn something new.~\cite{Doty2003}\\
This is applied in therapeutic storytelling, where through retelling experiences in story form, the storyteller attempts to better understand their own thoughts and situation. This can be supported by questions from a therapist who carefully steers the storyteller through their narration to pinpoint insights.~\cite{Lawless2001}\\
There are countless situations where we encounter storytelling precisely because we want to share experiences and emotions. Stories are used to inspire and motivate, to manage conflicts, for marketing or for political practice~\cite{Jameson2001}. These are all situations where the intent is separate from the story itself. Where we turn to storytelling because the human brain can process them so effectively. Another aspect is however when we tell stories for the story's sake.


\subsection{The Appeal of Stories}
I have now established that there is a difference between the intention of storytelling itself and utilizing it for another purpose. The difference is that we do not enter a courtroom to tell or listen to a story. We do so to find the truth. The past has simply show us that the careful recounting of events combined with precise inquiry, has proven a good way to do so. So when we do not tell stories as a means of achieving another intention, we also share them just because they are stories. For this we can take on both positions, that of the narrator or the listener. Again, a multitude of situations presents itself for these applications. We listen to bedtime stories that our elders tell us. We tell a funny anecdote to our friends on a night out. Humans have created whole industries around the consumption of stories. We read books to immerse ourselves into stories, we watch movies or shows and we play video games. The fact that so much of our time is willfully spent listening to, watching or interacting with stories, must mean that there is something worthwhile there. The mere consumption of a narration seems to be satisfying in itself.\\
When we hear stories, the brain releases the hormone Oxytocin. This hormone heightens feelings of trust, empathy and compassion. It positively influences social behavior and helps us to feel more connected to others.~\cite{Gottschall2012} Humans are inherently social beings. We want to connect to others around us. We try to create situations and use known cultural signals to connect on a non-verbal level. Humans mimic body language, laugh more in social situation than alone or use physical contact to communicate.~\cite{Frith2007}\\
When we hear a story, the brain is enabled to form connections to the people who listen to the story with us, to the narrator and to the characters in the story. Communication is a shared activity resulting in a transfer of information across brains. Research shows that during successful communication, the brains of both the speaker and the listener shows common, temporally linked, response activities. When we hear a story, our brain mirrors activities in the sensory center of the storyteller.~\cite{Stephens2010} This attempt to sync brain activity is a deeply social connection. It also means, that when we hear an enjoyable story, the brain behaves as if we would experience it ourselves.

\subsection{Stories in Media}

\subsection{Expectation \& Feedback}

\subsection{The optimal Experience}

\subsection{Quests}
% Story	- 3 pages
	% Appeal
	% Stories in media
	% Feedback - expectation
	% optimal experience
	% Quests
\section{Concretizing Information}
% Information - 3 pages
	% Types of Data
	% Mutation
\section{Game Mechanics}
% Game Mechanic - 3 pages
	% emergent
	% multiplicative game design
\section{Multi-Agent Systems}
% Agents - 3 pages
	% Games as multi agent systems
\section{Consensus}
% Consensus - 3 pages
	% Quorum
	% Proof of Personhood
\section{A Heuristic for Information}
% Heuristic for story relevant information 3 - pages
	% MapReduce
\section{Related Work}
\subsection{Emotion Engines}
% Related Work - Emotion Engines - 3 pages
\section{Inference engines}
% Inference engines - 3 pages