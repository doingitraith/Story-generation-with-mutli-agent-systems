\chapter{Comparison} %Comparison 4 pages
In order to properly evaluate the novelty and applicability of the presented game mechanic, it needs to be compared to different available games that employ similar game mechanics. This will help classify the game mechanic created during this thesis and how it holds up next to already released games.
\section{Comparison criteria}
There are many games that contain some form of dynamic storytelling, procedural narrative content, or emergent storytelling. Across the video game industry, those terms are not clearly defined and are sometimes used interchangeably or in a conflicting way. For that reason, while researching games where a comparison would be worthwhile, I decided on a set of criteria that I will use to differentiate the different games. The absence or presence of these criteria does, however, say nothing about the quality of a game, just whether it features a specific manifestation of dynamic storytelling.
\subsection{Reactiveness}
With these criteria, I evaluate whether a game changes depending on the actions of the player. It could be that narrative content is generated or remixed on every new game but does not actually react to anything the player does. This criterion should show, however, if the game’s narrative content does change as a consequence of player actions.
\subsection{Narrative Awareness}
Another important aspect is whether the change in narrative content is something that is acknowledged and reflected in the game or is the narrative change purely emergent and has to be interpreted by the player. Is the game aware of its change in narrative, or does it behave in such a way that the change is only apparent outside of it?
\subsection{Scripted Narrative}
When it comes to the narrative content itself, it is essential to consider how much of it is generated by game logic and how much of it is scripted beforehand by developers. This would include small bits of scripted narrative content that are reordered to create a new narrative.
\subsection{Replayability}
Finally, is the narrative generated in such a way that it would change when starting a new game, or will it be the same when the player does the same things? In other words, is the narrative generation deterministic or not?
\section{Games to compare}
To classify my own prototype in these criteria, I have selected several games that feature generated narrative content. Among the selected games are titles that have been available for several years, as well as games that were released very recently as of the writing of this thesis. I tried to select games that would all be considered in the area of \textit{generated narrative} or \textit{dynamic storytelling} but are very distinct and different examples of such narrative features. In the following, I will shortly describe the selected games and how they feature generated narrative content.
\subsection{Road 96}
In \textit{Road 96}~\cite{road}, the player takes on the role of multiple runaway teenagers who want to make their way across the border of a fictional country before an upcoming election might hinder them from leaving the country. The game is presented as a series of encounters with people who also find themselves on the road. It has only a small amount of gameplay systems like an energy bar or the tracking of money.\\
The heart of \textit{Road 96} is its multiple-choice dialogue segments with the different characters. At every new start of the game, the player will be put in the shoes of a different hitchhiker and meet different people from a recurring cast of characters. Environmental details are changed every time, and the dialogue options are dependent on the characters and whom they meet. \textit{Road 96} is a very complex choice tree that the player traverses as characters encounter each other. The choices players make during the game sessions will influence the ending of the individual character. All possible combinations of characters are, however, previously written and finite, yet exhaustive.
\subsection{Crusader Kings III}
Paradox's \textit{Crusader Kings III}~\cite{crusader} is a political strategy and simulation game set between 867 and 1453 across Europe, Asia, and Central Africa. The player controls the heads of their dynasty across many generations who have character traits and skills that develop over the game. Political choices such as trading, warfare, and diplomatic interactions change the political landscape and create complex game states.\\
The game has very complex, interlocking systems, including several government types, a genetic system for passing down character traits to descendants, a social system for interacting with other characters, and many more. This leads to incredibly varied and complex narrative scenarios because the different systems interact with each other.
\subsection{Divinity: Original Sin II - Game Master Mode}
Part of \textit{Divinity: Original Sin II}~\cite{divinity} is a game mode that is still quite unique in the video game landscape - the \textit{Game Master Mode}. The base game \textit{Divinity: Original Sin 2} is a critically acclaimed fantasy role-playing game that is similar to genre contemporaries that are based on tabletop games like \textit{Dungeons and Dragons}. Players create a character with a set of skills and stats that determine what they can do in a game, how they fight, and how they interact with other characters in the game.\\
The addition of the \textit{Game Master Mode} now allows players to create content within the context of \textit{Divinity: Original Sin 2}. It is designed to be played as an asymmetrical multiplayer game where one player takes on the role of the \textit{Game Master}. This role opens up the game engine and lets the player access a content creation toolkit. They then can create narrative scenes, place objects and characters in the 3D environments and prepare encounters, quests, and interactions. When players join such a scenario, they can then play through it interactively and more in the style of a conventional tabletop experience. The players control their characters and use game mechanics and systems from \textit{Divinity: Original Sin 2} but experience a story that is entirely user-created. There are additional tools available to the \textit{Game Master} that enable player groups to react to situations that exceed the limitations of the base game. Additional rules can be created or brought into the game from other role-playing game systems.\\
This means that the narrative content is not technically \textit{generated} by the game, but it is highly adaptive and more variable because it introduces an actual human narrator.
\subsection{Darkest Dungeon}
When released in 2016, \textit{Darkest Dungeon}~\cite{dungeon} was highly praised for its rich atmosphere and immersive setting. In the role-playing game, the player leads a party of randomly generated adventurer characters into procedurally generated dungeons. There, they encounter threats and obstacles in the form of enemies and negative character modifiers like mental disorders and other emotional traumas. The game focuses on the psychological toll adventures take on the protagonists and presents a sinister and dark fantasy world.\\
Although there is an overarching plot in the game, it is only dealt out in tiny bits in order to tie the many dungeon expeditions together. The expeditions themselves are presented as "mini-stories" in which characters venture into the unknown and encounter terrible dangers that leave them scarred forever. A narrator's voice adds atmospheric one-liners that underline the characters' struggles and weaves a narrative throughline into each expedition.
\subsection{Unexplored 2: The Wayfarer's Legacy}
A very recent title is \textit{Unexplored 2: The Wayfarer's Legacy}~\cite{unexplored}. It is another role-playing game that uses its interacting gameplay systems to create emergent narrative content. A created character is tasked with destroying a magical object that could allow an evil force to conquer the world. Players have to explore, fight, and talk their way through different challenges. Player actions however, do not happen in a vacuum and changes are persistent even if a character dies. The world does not reset and randomize itself. Each death moves the clock forward a few years. During those intervening years, the antagonist party's influence expands and makes the journey more dangerous. Alliances between clans strengthen or dissolve, and new quests replace old ones. Although the map stays the same, the layout, enemies, and challenges found in each individual area change. When it makes sense, though, certain things persist in between games. For example, unique objects gained in one game and then lost because of death could be recovered by the next character at that exact location or might have been taken away by an NPC.\\
There is not much of a main story present in \textit{Unexplored 2}, but all of these complicated systems interlock in ways that organically create a new one with each character. Even the skill check system tells little stories. That is frequently used to determine everything from dialogue resolution to solving stat-based puzzles where success is partially influenced by the character's skills.
\subsection{Prototype Game}
The prototype game developed as part of this thesis has been described in great detail already. I just want to recap its narrative properties in the context of this classification.\\
The prototype game is modeled as a multi-agent system that allows the exchange, retrieval, and introduction of information with other agents. There is no built-in AI that tries to create storytelling behavior, but the designers can create quests that are defined through rules that need to be satisfied. The goal of the prototype is to create an emergent narrative through the interaction of its systems.
\section{Results}
I have analyzed several games through the lens of the described classification criteria. The selected games all feature mechanics that fall under the umbrella term of \textit{dynamic storytelling} but apply the creation of narrative content differently.\\
The final result of the classification shows that the games all behave in different ways when replayed multiple times. This is not yet true for the prototype game since the game does not yet properly react to the player (Table~\ref{table:comparison}). The generated narrative is purely emergent.\\
All the considered games except for \textit{Road 96} and the prototype game actually react to player actions. Games like \textit{Road 96} follow a more "choose your own adventure" approach where all options are known upfront and are then mixed and matched to create new variations.\\
Three of the analyzed titles, \textit{Road 96}, \textit{Crusader Kings III}, and the \textit{Game Master Mode} in \textit{Divinity: Original Sin 2}, have capabilities to generate narrative content that follows a dramatic structure. They do so differently, though by reordering encounters in a more dramatic fashion, using the game AI to let characters behave in specific ways, or by introducing a human narrator to steer the storyline.\\
It is worth noting that all games except for \textit{Crusader Kings III} contain some form of existing storyline or narrative frame in which the game takes place. Only \textit{Crusader Kings III} lets its gameplay systems alone in combinations with the player actions dictate the unfolding, fully emergent story. 
\begin{table}[h!]
\centering
\resizebox{\textwidth}{!}{
\begin{tabular}{|p{3.5cm}||c|c|c|c|}
\hline
\textbf{Game} & \textbf{Reactiveness} & \textbf{Narrative Awareness} & \textbf{Scripted Narrative} & \textbf{Replayability}\\
\hline \hline
\textbf{Road 96} & No & Yes & Yes & Yes \\
\hline
\textbf{Crusader Kings III} & Yes & Yes & No & Yes \\
\hline
\textbf{Divinity: Original Sin 2 - Game Master Mode} & Yes & Yes & No & Yes \\
\hline
\textbf{Darkest Dungeon} & Yes & No & Yes & Yes \\
\hline
\textbf{Unexplored II: The Wayfarer's Legacy} & Yes & No & Yes & Yes \\
\hline
\textbf{Prototype Game} & No & No & Yes & No \\
\hline
\end{tabular}}
\caption{Results of game comparison}
\label{table:comparison}
\end{table}